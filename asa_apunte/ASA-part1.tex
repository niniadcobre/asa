
\section{Objetivos}
\subsection{De la carrera}
Según el documento fundamental de la Tecnicatura, el o la Técnico/a Superior en Administración de Sistemas y Software Libre estará capacitado/a para:
\begin{itemize}
	\item Desarrollar actividades de administración de infraestructura. Comprendiendo la administración de sistemas, redes y los distintos componentes que forman la
infraestructura de tecnología de una institución, ya sea pública o privada.
	\item Aportar criterios básicos para la toma de decisiones relativas a la adopción de nuevas tecnologías libres.
	\item Desempeñarse como soporte técnico, solucionando problemas afines por medio de la comunicación con comunidades de Software Libre, empresas y desarrolladores de
software.
	\item Realizar tareas de trabajo en modo colaborativo, intrínseco al uso de tecnologías libres.
	\item Comprender y adoptar el estado del arte local, nacional y regional en lo referente a implementación de tecnologías libres. Tanto en los aspectos técnicos como legales.
\end{itemize}
\subsection{De la asignatura}

\begin{itemize}
	\item Saber implementar configuraciones especiales de almacenamiento
	\item Saber aplicar programación avanzada a la automatización de tareas
	\item Saber diseñar e implementar estrategias de respaldo 
	\item Conocer formas de implementar estrategias de tolerancia a fallos para servicios críticos
\end{itemize}


\section {Contenidos}
\subsection{Contenidos mínimos}
\begin{itemize}
	\item  Instalación sobre configuraciones de almacenamiento especiales. 
	\item  Scripting avanzado. 
	\item  Planificación de tareas. 
	\item  Virtualización. 
	\item  Alta Disponibilidad.
\end{itemize}


\subsection {Programa}
\begin{enumerate}
\item Configuraciones de almacenamiento
\begin{itemize}
	\item   Arquitectura de E/S, Dispositivos de E/S, Filesystems
	\item	Diseños típicos de almacenamiento
	\item	Software RAID, instalación y mantenimiento niveles 0, 1, 10, 5 
	\item	LVM, instalación y mantenimiento	 
\end{itemize}
	
\item Estrategias de respaldo
\begin{itemize}
	\item Copiado y sincronización de archivos
	\item Estrategias y herramientas de backup, LVM snapshots
	\item Control de versiones
\end{itemize}

\item Alta Disponibilidad
\begin{itemize}
	\item Clustering de LB, de HA, de HPC. Conceptos de HA.
	\item Peacemaker/Corosync, DRBD, Clustering de aplicaciones
\end{itemize}

\item Virtualización
\begin{itemize}
	\item Formas de virtualización, herramientas. 
	\item Creación, instalación, migración, eliminación de MV y container.
        \item Container y VM.  
	\item Concepto de Cloud. IaaS, PaaS, SaaS, etc.
\end{itemize}
\item Scripting avanzado
\begin{itemize}
	\item Accesibilidad. Interfaz gráfica. 
	\item Captura de señales: trap. 
\end{itemize}
\end{enumerate}

\section {Bibliografía inicial}
\begin{itemize}
	\item Kemp, Juliet. Linux System Administration Recipes: A Problem-Solution Approach. Apress, 2009. 
	\item Lakshman, Sarath. Linux Shell Scripting Cookbook Solve Real-World Shell Scripting Problems with over 110 Simple but Incredibly Effective Recipes. Birmingham, U.K.: Packt Pub., 2011. 
	\item Parker, Steve. Shell Scripting Expert Recipes for Linux, Bash, and More. Hoboken, N.J.; Chichester: Wiley; John Wiley, 2011.
	\item K. Kopper, The Linux Enterprise Cluster: build a highly available cluster with commodity hardware and free software. San Francisco: No Starch Press, 2005.
	\item Andrew Beekhof, Pacemaker 1.1 Cluster from scratch. 2016.
	\item R. Pollei, Debian 7 System Administration Best Practices. Birmingham: Packt Publishing, 2013.
	\item T. A. Limoncelli, C. J. Hogan, and S. R. Chalup, The practice of system and network administration. Upper Saddle River, N.J: Addison-Wesley, 2008.
	\item Chen, Peter M., et al., RAID: High-performance, reliable secondary storage. ACM Computing Surveys (CSUR), 1994.
	\item S. van Vugt, Pro Linux high availability clustering. 2014.


\end{itemize}

\subsection{Biblioteca Virtual del MinCyT}

Biblioteca Virtual del MinCyT: \url{http://www.biblioteca.mincyt.gob.ar/libros}.

Títulos accesibles desde la UNC:

\begin{itemize}
	\item C. Wolf and E. M. Halter, Virtualization from the desktop to the enterprise. Berkeley, CA; New York, NY: Apress; Distributed in U.S. by Springer-Verlag New York, 2005; \url{http://rd.springer.com/book/10.1007/978-1-4302-0027-7}.
	\item K. Schmidt, High availability and disaster recovery concepts, design, implementation. Berlin; Springer, 2006; \url{http://rd.springer.com/book/10.1007/3-540-34582-5}.	
\end{itemize}
