\section{Opciones de rsync}
\label{sec:rsync}
\begin{lstlisting}
$ rsync --help
rsync  version 3.1.0  protocol version 31
Copyright (C) 1996-2013 by Andrew Tridgell, Wayne Davison, and others.
Web site: http://rsync.samba.org/
Capabilities:
    64-bit files, 64-bit inums, 32-bit timestamps, 64-bit long ints,
    socketpairs, hardlinks, symlinks, IPv6, batchfiles, inplace,
    append, ACLs, xattrs, iconv, symtimes, prealloc

rsync comes with ABSOLUTELY NO WARRANTY.  This is free software, and you
are welcome to redistribute it under certain conditions.  See the GNU
General Public Licence for details.

rsync is a file transfer program capable of efficient remote update
via a fast differencing algorithm.

Usage: rsync [OPTION]... SRC [SRC]... DEST
  or   rsync [OPTION]... SRC [SRC]... [USER@]HOST:DEST
  or   rsync [OPTION]... SRC [SRC]... [USER@]HOST::DEST
  or   rsync [OPTION]... SRC [SRC]... rsync://[USER@]HOST[:PORT]/DEST
  or   rsync [OPTION]... [USER@]HOST:SRC [DEST]
  or   rsync [OPTION]... [USER@]HOST::SRC [DEST]
  or   rsync [OPTION]... rsync://[USER@]HOST[:PORT]/SRC [DEST]
The ':' usages connect via remote shell, while '::' & 'rsync://' usages connect
to an rsync daemon, and require SRC or DEST to start with a module name.

Options
 -v, --verbose               increase verbosity
     --info=FLAGS            fine-grained informational verbosity
     --debug=FLAGS           fine-grained debug verbosity
     --msgs2stderr           special output handling for debugging
 -q, --quiet                 suppress non-error messages
     --no-motd               suppress daemon-mode MOTD (see manpage caveat)
 -c, --checksum              skip based on checksum, not mod-time & size
 -a, --archive               archive mode; equals -rlptgoD (no -H,-A,-X)
     --no-OPTION             turn off an implied OPTION (e.g. --no-D)
 -r, --recursive             recurse into directories
 -R, --relative              use relative path names
     --no-implied-dirs       don't send implied dirs with --relative
 -b, --backup                make backups (see --suffix & --backup-dir)
     --backup-dir=DIR        make backups into hierarchy based in DIR
     --suffix=SUFFIX         set backup suffix (default ~ w/o --backup-dir)
 -u, --update                skip files that are newer on the receiver
     --inplace               update destination files in-place (SEE MAN PAGE)
     --append                append data onto shorter files
     --append-verify         like --append, but with old data in file checksum
 -d, --dirs                  transfer directories without recursing
 -l, --links                 copy symlinks as symlinks
 -L, --copy-links            transform symlink into referent file/dir
     --copy-unsafe-links     only "unsafe" symlinks are transformed
     --safe-links            ignore symlinks that point outside the source tree
     --munge-links           munge symlinks to make them safer (but unusable)
 -k, --copy-dirlinks         transform symlink to a dir into referent dir
 -K, --keep-dirlinks         treat symlinked dir on receiver as dir
 -H, --hard-links            preserve hard links
 -p, --perms                 preserve permissions
 -E, --executability         preserve the file's executability
     --chmod=CHMOD           affect file and/or directory permissions
 -A, --acls                  preserve ACLs (implies --perms)
 -X, --xattrs                preserve extended attributes
 -o, --owner                 preserve owner (super-user only)
 -g, --group                 preserve group
     --devices               preserve device files (super-user only)
     --specials              preserve special files
 -D                          same as --devices --specials
 -t, --times                 preserve modification times
 -O, --omit-dir-times        omit directories from --times
 -J, --omit-link-times       omit symlinks from --times
     --super                 receiver attempts super-user activities
     --fake-super            store/recover privileged attrs using xattrs
 -S, --sparse                handle sparse files efficiently
     --preallocate           allocate dest files before writing them
 -n, --dry-run               perform a trial run with no changes made
 -W, --whole-file            copy files whole (without delta-xfer algorithm)
 -x, --one-file-system       don't cross filesystem boundaries
 -B, --block-size=SIZE       force a fixed checksum block-size
 -e, --rsh=COMMAND           specify the remote shell to use
     --rsync-path=PROGRAM    specify the rsync to run on the remote machine
     --existing              skip creating new files on receiver
     --ignore-existing       skip updating files that already exist on receiver
     --remove-source-files   sender removes synchronized files (non-dirs)
     --del                   an alias for --delete-during
     --delete                delete extraneous files from destination dirs
     --delete-before         receiver deletes before transfer, not during
     --delete-during         receiver deletes during the transfer
     --delete-delay          find deletions during, delete after
     --delete-after          receiver deletes after transfer, not during
     --delete-excluded       also delete excluded files from destination dirs
     --ignore-missing-args   ignore missing source args without error
     --delete-missing-args   delete missing source args from destination
     --ignore-errors         delete even if there are I/O errors
     --force                 force deletion of directories even if not empty
     --max-delete=NUM        don't delete more than NUM files
     --max-size=SIZE         don't transfer any file larger than SIZE
     --min-size=SIZE         don't transfer any file smaller than SIZE
     --partial               keep partially transferred files
     --partial-dir=DIR       put a partially transferred file into DIR
     --delay-updates         put all updated files into place at transfer's end
 -m, --prune-empty-dirs      prune empty directory chains from the file-list
     --numeric-ids           don't map uid/gid values by user/group name
     --usermap=STRING        custom username mapping
     --groupmap=STRING       custom groupname mapping
     --chown=USER:GROUP      simple username/groupname mapping
     --timeout=SECONDS       set I/O timeout in seconds
     --contimeout=SECONDS    set daemon connection timeout in seconds
 -I, --ignore-times          don't skip files that match in size and mod-time
 -M, --remote-option=OPTION  send OPTION to the remote side only
     --size-only             skip files that match in size
     --modify-window=NUM     compare mod-times with reduced accuracy
 -T, --temp-dir=DIR          create temporary files in directory DIR
 -y, --fuzzy                 find similar file for basis if no dest file
     --compare-dest=DIR      also compare destination files relative to DIR
     --copy-dest=DIR         ... and include copies of unchanged files
     --link-dest=DIR         hardlink to files in DIR when unchanged
 -z, --compress              compress file data during the transfer
     --compress-level=NUM    explicitly set compression level
     --skip-compress=LIST    skip compressing files with a suffix in LIST
 -C, --cvs-exclude           auto-ignore files the same way CVS does
 -f, --filter=RULE           add a file-filtering RULE
 -F                          same as --filter='dir-merge /.rsync-filter'
                             repeated: --filter='- .rsync-filter'
     --exclude=PATTERN       exclude files matching PATTERN
     --exclude-from=FILE     read exclude patterns from FILE
     --include=PATTERN       don't exclude files matching PATTERN
     --include-from=FILE     read include patterns from FILE
     --files-from=FILE       read list of source-file names from FILE
 -0, --from0                 all *-from/filter files are delimited by 0s
 -s, --protect-args          no space-splitting; only wildcard special-chars
     --address=ADDRESS       bind address for outgoing socket to daemon
     --port=PORT             specify double-colon alternate port number
     --sockopts=OPTIONS      specify custom TCP options
     --blocking-io           use blocking I/O for the remote shell
     --stats                 give some file-transfer stats
 -8, --8-bit-output          leave high-bit chars unescaped in output
 -h, --human-readable        output numbers in a human-readable format
     --progress              show progress during transfer
 -P                          same as --partial --progress
 -i, --itemize-changes       output a change-summary for all updates
     --out-format=FORMAT     output updates using the specified FORMAT
     --log-file=FILE         log what we're doing to the specified FILE
     --log-file-format=FMT   log updates using the specified FMT
     --password-file=FILE    read daemon-access password from FILE
     --list-only             list the files instead of copying them
     --bwlimit=RATE          limit socket I/O bandwidth
     --outbuf=N|L|B          set output buffering to None, Line, or Block
     --write-batch=FILE      write a batched update to FILE
     --only-write-batch=FILE like --write-batch but w/o updating destination
     --read-batch=FILE       read a batched update from FILE
     --protocol=NUM          force an older protocol version to be used
     --iconv=CONVERT_SPEC    request charset conversion of filenames
     --checksum-seed=NUM     set block/file checksum seed (advanced)
 -4, --ipv4                  prefer IPv4
 -6, --ipv6                  prefer IPv6
     --version               print version number
(-h) --help                  show this help (-h is --help only if used alone)

Use "rsync --daemon --help" to see the daemon-mode command-line options.
Please see the rsync(1) and rsyncd.conf(5) man pages for full documentation.
See http://rsync.samba.org/ for updates, bug reports, and answers
\end{lstlisting}

%\section{Ejemplo completo de configuración DRBD}
%\label{sec:confdrbd}
%
%\begin{lstlisting}
%resource example {
%	options {
%		on-no-data-accessible suspend-io;
%	}
%
%	net {
%		cram-hmac-alg "sha1";
%		shared-secret "secret_string";
%	}
%
%	# The disk section is possible on resource level and in each
%	# volume section
%	disk {
%		# If you have a resonable RAID controller
%		# with non volatile write cache (BBWC, flash)
%		disk-flushes no;
%		disk-barrier no;
%		md-flushes no;
%	}
%
%	# volume sections on resource level, are inherited to all node
%	# sections. Place it here if the backing devices have the same
%	# device names on all your nodes.
%	volume 1 {
%		device minor 1;
%		disk /dev/sdb1;
%		meta-disk internal;
%
%		disk {
%			resync-after example/0;
%		}
%	}
%
%	on wurzel {
%		address	192.168.47.1:7780;
%
%		volume 0 {
%		       device minor 0;
%		       disk /dev/vg_wurzel/lg_example;
%		       meta-disk /dev/vg_wurzel/lv_example_md;
%		}
%	}
%	on sepp {
%		address	192.168.47.2:7780;
%
%		volume 0 {
%		       device minor 0;
%		       disk /dev/vg_sepp/lg_example;
%		       meta-disk /dev/vg_sepp/lv_example_md;
%		}
%	}
%}
%
%resource "ipv6_example_res" {
%	net {
%		cram-hmac-alg "sha1";
%		shared-secret "ieho4CiiUmaes6Ai";
%	}
%
%	volume 2 {
%		device	"/dev/drbd_fancy_name" minor 0;
%		disk	/dev/vg0/example2;
%		meta-disk internal;
%	}
%
%	on amd {
%		# Here is an example of ipv6.
%		# If you want to use ipv4 in ipv6 i.e. something like [::ffff:192.168.22.11]
%		# you have to set disable-ip-verification in the global section.
%		address	ipv6 [fd0c:39f4:f135:305:230:48ff:fe63:5c9a]:7789;
%	}
%
%	on alf {
%		address ipv6 [fd0c:39f4:f135:305:230:48ff:fe63:5ebe]:7789;
%	}
%}
%
%
%#
%# A two volume setup with a node for disaster recovery in an off-site location.
%#
%
%resource alpha-bravo {
%	net {
%		cram-hmac-alg "sha1";
%		shared-secret "Gei6mahcui4Ai0Oh";
%	}
%
%	on alpha {
%		volume 0 {
%			device minor 0;
%			disk /dev/foo;
%			meta-disk /dev/bar;
%		}
%		volume 1 {
%			device minor 1;
%			disk /dev/foo1;
%			meta-disk /dev/bar1;
%		}
%		address	192.168.23.21:7780;
%	}
%	on bravo {
%		volume 0 {
%			device minor 0;
%			disk /dev/foo;
%			meta-disk /dev/bar;
%		}
%		volume 1 {
%			device minor 1;
%			disk /dev/foo1;
%			meta-disk /dev/bar1;
%		}
%		address	192.168.23.22:7780;
%	}
%}
%
%resource stacked_multi_volume {
%	net {
%		protocol A;
%
%		on-congestion pull-ahead;
%		congestion-fill 400M;
%		congestion-extents 1000;
%	}
%
%	disk {
%		c-fill-target 10M;
%	}
%
%	volume 0 { device minor 10; }
%	volume 1 { device minor 11; }
%
%	proxy {
%		memlimit 500M;
%		plugin {
%			lzma contexts 4 level 9;
%		}
%	}
%
%	stacked-on-top-of alpha-bravo {
%		address	192.168.23.23:7780;
%
%		proxy on charly {
%			# In the regular production site, there is a dedicated host to run
%			# DRBD-proxy
%			inside    192.168.23.24:7780; # for connections to DRBD
%			outside   172.16.17.18:7780; # for connections over the WAN or VPN
%			options {
%				memlimit 1G; # Additional proxy options are possible here
%			}
%		}
%	}
%	on delta {
%		volume 0 {
%			device minor 0;
%			disk /dev/foo;
%			meta-disk /dev/bar;
%		}
%		volume 1 {
%			device minor 1;
%			disk /dev/foo1;
%			meta-disk /dev/bar1;
%		}
%		address	127.0.0.2:7780;
%
%		proxy on delta {
%			# In the DR-site the proxy runs on the machine that stores the data
%			inside 127.0.0.1:7780;
%			outside 172.16.17.19:7780;
%		}
%	}
%}
%
%resource drbd_9_two_connection {
%	volume 0 {
%	       device minor 10;
%	       disk /dev/foo/bar;
%	       meta-disk internal;
%	}
%
%	on alpha {
%		node-id 0;
%		address 192.168.31.1:7800;
%	}
%	on bravo {
%		node-id 1;
%		address 192.168.31.2:7800;
%	}
%	on charlie {
%		node-id 2;
%		address 192.168.31.3:7800;
%	}
%
%	net {
%		ko-count 3;
%	}
%
%	connection "optional name" {
%		host alpha;
%		host bravo;
%		net { protocol C; }
%	}
%
%	connection {
%		host alpha address 127.0.0.1:7800 via proxy on alpha {
%			inside 127.0.0.2:7800;
%			outside 192.168.31.1:7801;
%		}
%		host charlie address 127.0.0.1:7800 via proxy on charlie {
%			inside 127.0.0.2:7800;
%			outside 192.168.31.3:7800;
%		}
%		net { protocol A; }
%	}
%
%	connection {
%		host bravo address 127.0.0.1:7800 via proxy on bravo {
%			inside 127.0.0.2:7800;
%			outside 192.168.31.2:7801;
%		}
%		host charlie address 127.0.0.1:7800 via proxy on charlie {
%			inside 127.0.0.2:7800;
%			outside 192.168.31.3:7800;
%		}
%		net { protocol A; }
%	}
%}
%
%resource drbd_9_mesh {
%	volume 0 {
%	       device minor 11;
%	       disk /dev/foo/bar2;
%	       meta-disk internal;
%	}
%
%	on alpha {
%		node-id 0;
%		address 192.168.31.1:7900;
%	}
%	on bravo {
%		node-id 1;
%		address 192.168.31.2:7900;
%	}
%	on charlie {
%		node-id 2;
%		address 192.168.31.3:7900;
%	}
%
%	connection-mesh {
%		hosts alpha bravo charlie;
%		net {
%			protocol C;
%		}
%	}
%}
%\end{lstlisting}
